\documentclass[a4paper]{article}


\title{Conservation and allowed residues}
\author{Andrew C.R.\ Martin}
\date{21st December 2022}

\begin{document}
\maketitle

\section{Introduction}
We have previously looked at the problem of conservation in columns in
a multiple sequence alignment (MSA), and generally have ways of
scoring conservation including our version of `scorecons'. We have
also looked at weighting the conservation by species-level sequence
conservation and have implemented a statistical cutoff for significant
conservation in an alignment-dependent manner (ImPACT).

However, this does not address the problem of whether a
\emph{specific} mutation is allowed. Suppose we have 100 sequences
where 90 of them have a serine at a given position and the remaining
10 have a threonine. Since serine and threonine are very similar amino
acids and score highly in things like BLOSUM or PAM scoring matrices,
this will be indicated as a very conserved position. Using our
`scorecons' and the default pet91 matrix, this would obtain a
conservation score of 0.891. Using the grouped entropy scoring, it
obtains a score of 1.0.

Now suppose we have a native sequence having a serine at this position,
if a mutation is made to a very different amino acid then this is very
likely to be problematic. However, if the mutation is to a serine, on
the basis of conservation, this will still be scored as problematic
although we know that this is a very similar amino acid and is allowed
in other sequences.

A further problem is the concept of `compensated pathogenic mutations'
(CPDs). Here a mutation known to be pathogenic in (say) humans results
in the native residue found in another species (say mice). Clearly
there are other accompanying mutations that compensate for this
change. Thus the mutation in our species of interest is damaging, yet
the resulting mutant residue is seen in an MSA at this position.

\section{Pre-requisites}
The following pre-requisites and assumptions are made below:

\begin{enumerate}
  \item Any scoring matrix may be used, but it must be normalized such
    that a)~all values are positive, b)~the values on the diagonal are
    equal and maximal, c)~the values on the diagonal are equal to
    1. The Valdar scorecons paper explains how to do this.
  \item An MSA is available for Proposals~4 and~5.
  \item A method is available for scoring sequence identity or
    sequence similarity (between two pre-aligned sequences) for
    Proposals~4 and~5. In both cases, the value should be expressed as
    a number between 0 and~1.
\end{enumerate}

\section{Proposal~1 --- Conservation with respect to a reference
  sequence} 
The first proposal is that, instead of calculating an overall
conservation score for a position in an alignment, we should only
look at conservation in a pairwise manner from a reference sequence of
interest. In `scorecons' we calculate a score for a particular column as:
\begin{equation}
  S_c = \sum_{i, j>i}^Ns_{i, j} / N_c
\end{equation}
where $S_c$ is the score for this column, $s_{i,j}$ is a score for
residues at positions $i$ and $j$ in the column, maximal and
equal to $1$), $N$ is the number of sequences in the alignment and
$N_c$ is the number of comparisons made ($N_c = N(N-1)/2$).

Instead, we would calculate:
\begin{equation}\label{eqn:sr}
  S_{r} = \sum_{j>1}^Ns_{1, j} / (N-1)
\end{equation}
i.e.\ $S_r$ is simply the conservation score \emph{with respect to
  the reference sequence}.

\section{Proposal~2 --- minimum conservation}
The second proposal is that, instead of calculating the mean
conservation, we should calculate the minimum score for a substituted
residue. In other words what is the minimum conservation (i.e.\
maximum diversity) allowed at this position?

\begin{equation} \label{eqn:sm}
  S_{m} = \min_{j>1}^Ns_{1, j}
\end{equation}
This could then be used as a threshold for suggesting whether a
particular mutation is likely to be damaging. i.e.\ we would calculate
the difference between a score for a given mutation from the native
sequence and $S_m$:

\begin{equation}
  \Delta_m = s_{1, m} - S_m
\end{equation}
where $s_{1,m}$ is the score for mutated residue $m$ against the
reference sequence residue and $S_m$ is the thresold calculated in
Equation~\ref{eqn:sm}. A negative value would indicate an unfavourable
mutation. 

\section{Proposal~3 --- minimum conservation by S.D.}
Proposal~3 refines Proposal~2 by selecting a score that is $d$
standard deviations lower than the mean rather than simply selecting
the minimum. This accounts for the distribution of mutated residues
rather than simply taking the most different. The standard deviation
is calculated as:

\begin{equation}\label{eqn:sigma}
  \sigma = \sqrt{\frac{\sum_{j>1}^N(s_{1,j} - S_r)^2}{N-1}}
\end{equation}
where $S_r$ is the mean conservation with respect to the reference
sequence as calculated in Equation~\ref{eqn:sr}. A threshold score would
then be calculated as:
\begin{equation}\label{eqn:ssigma}
  S_\sigma = S_r - d\sigma
\end{equation}
where $d$ would have to be determined empirically, but typically could
be $2.5$ or $3.0$.

As before, we could then calculate the difference between a score for
a given mutation from the native sequence and $S_\sigma$:

\begin{equation}
  \Delta_\sigma = s_{1, m} - S_\sigma
\end{equation}
where $s_{1,m}$ is the score for mutated residue $m$ against the
reference sequence residue and $S_\sigma$ is the thresold calculated in
Equation~\ref{eqn:ssigma}. A negative value would indicate an unfavourable
mutation. 


\section{Proposal~4 --- weighted minimum conservation}
Proposal~4 attempts to allow for CPDs. CPDs become more common as the
sequences become more diverse. i.e.\ it is often small cummulative
effects rather than a single compensatory mutation (see our paper).
Thus the idea would be to weight allowed diversity higher for
sequences that are overall very similar and lower for sequences that
are less similar.

For example, suppose the reference sequence has a serine at a
particular position and another sequence is identical except for an
asparagine at that position. Clearly that asparagine is going to be
allowed as a mutation with no effect on the function of the protein
(unless the compensatory mutation is in another protein). On the other
hand, if two sequences are quite diverse, there may be compensatory
effects going on.

In this case, it is simpler to work in terms of diversity rather than
conservation of residues. Assuming (as above) that residue similarity
from a scoring matrix is scored in the range of 0\ldots 1 (where 1
represents an identical residue), we can simply define the diversity
($v$) as:

\begin{equation}\label{eqn:v}
  v_{i,j} = 1 - s_{i,j}
\end{equation}

We can now revisit Proposal~2, weighting by the sequence identity (or
similarity --- both expressed as a fractional value, $P$) between the
two sequences:

\begin{equation} \label{eqn:smw}
  V_{mw} = \max_{j>1}^N(v_{1, j} \times P_{1,j})
\end{equation}
where $v_{1,j}$ is the diversity between the residue in the reference
sequence residue and in sequence $j$ as defined in
Equation~\ref{eqn:v} and $P_{1,j}$ is the overall fractional sequence
identity (or similarity) between sequences the reference sequence and
sequence $j$.  As before, this could then be used as a threshold for
suggesting whether a particular mutation is likely to be damaging:

\begin{equation}
  \Delta_{mw} = V_{mw} - (v_{1, M} \times P_{1, M}) 
\end{equation}
where $v_{1,M}$ is the diversity score for mutated residue $M$ against
the reference sequence residue and $V_{mw}$ is the threshold
calculated in Equation~\ref{eqn:smw}. $P_{1,M}$ is the sequence
identity (or similarity) between the reference sequence and the
mutated sequence (which clearly will be very high, normally only one
mutation being present). A negative value would indicate an
unfavourable mutation.

\section{Proposal~5 --- weighted minimum conservation by SD}
Proposal~5 is simply the ideas of Proposal~4 applied to Proposal~3
instead of Proposal~2, i.e.\ selecting a diversity score that is $d$ standard
deviations above the mean rather than simply selecting the
maximum.

The mean weighted diversity score is calculated with resepct to the
reference sequence (analagous to Equation~\ref{eqn:sr}):

\begin{equation}\label{eqn:vr}
  V_{r} = \sum_{j>1}^N(v_{1, j} \times P_{1,j}) / (N-1)
\end{equation}
The standard deviation is calculated as (analagous to Equation~\ref{eqn:sigma}):

\begin{equation}
  \sigma_w = \sqrt{\frac{\sum_{j>1}^N((v_{1,j} \times P_{1,j}) - V_r)^2}{N-1}}
\end{equation}
A threshold score would then be calculated as:
\begin{equation}\label{eqn:ssigmaw}
  V_{\sigma w} = V_r + d\sigma_w
\end{equation}
where $d$ would have to be determined empirically, but typically could
be $2.5$ or $3.0$.

As before, we could then calculate the difference between a score for
a given mutation from the native sequence and $S_{\sigma w}$:

\begin{equation}
  \Delta_{\sigma w} = V_{\sigma w} - (v_{1, M} \times P_{1, M})
\end{equation}
where $v_{1,M}$ is the diversity score for mutated residue $M$ against
the reference sequence residue, $V_{\sigma w}$ is the thresold
calculated in Equation~\ref{eqn:ssigmaw} and $P_{1,M}$ is the sequence
identity (or similarity) between the reference sequence and the
mutated sequence (which clearly will be very high, normally only one
mutation being present). A negative value would indicate an
unfavourable mutation.

\end{document}
